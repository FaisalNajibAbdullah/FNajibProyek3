\chapter{PENDAHULUAN}


\section{Latar Belakang}
\par Meningkatnya jumlah penduduk menyebabkan bertambah pula kepemilikan kendaraan pribadi\cite{supriono2015evaluasi}. Hal ini menimbulkan perlunya lahan parkir akan semakin besar, terutama di daerah perkotaan yang padat \cite{supriono2015evaluasi}. Untuk memenuhi kebutuhan tersebut maka pada  kawasan-kawasan tertentu disediakan tempat parkir. Tempat parkir merupakan tempat dimana pemilik kendaraan menghentikan kendaraannya dalam jangka waktu tertentu sesuai dengan kebutuhan pemilik kendaraan \cite{kurniawan2017analisis}. Tempat parkir biasanya dapat berupa gedung parkir dan taman parkir\cite{kurniawan2017analisis}.
\par Namun permasalahan akan timbul ketika tempat parkir tersebut sangat luas areanya, apalagi tempat parkir yang berupa gedung yang tentunya memiliki beberapa lantai yang menyebabkan pemilik kendaraan kesulitan dalam mencari slot parkir yang kosong dan mau tidak mau pemilik kendaraan berkeliling terlebih dahulu untuk mencarinya\cite{danisia2017prototipe}. Hal ini akan membuang waktu si pemilik kendaraan. Keamanan dari tempat parkir tersebut juga belum tentu terjamin, sehingga memungkinkan terjadinya pencuriaan kendaraan\cite{fais2014pengembangan}. Untuk memudahkan kita mencari tempat parker tanpa perlu menghabiskan waktu kita dengan berputar putar mencari tempat yang kosong. Dengan teknologi hari ini memungkinkan kita dapat mencari tempat parkir yang kosong tanpa menghabiskan waktu kita, dengan suatu alat yang terhubung dengan kamera cctv yang akan menginformasikan dimana tempat yang kosong langsung ke dalam aplikasi berbasis mobile\cite{fraifer2016designing}.
\par Sebuah alat berbasis IoT dengan menggunakan metode Opencv yang nantinya bisa memonitor slot parkir mana yang masih kosong. Nantinya data dari alat tersebut akan ditampilkan pada monitor yang telah disediakan di tempat pengambilan tiket parkir atau ditampilkan pada aplikasi berbasis mobile. Sehingga nantinya pemilik kendaraan tidak perlu repot-repot untuk mencari slot parkir yang kosong dan juga pemilik kendaraan bisa melihat letak parkir kendaraannya jika lupa melalui gawai masing-masing serta pemilik kendaraan bisa memonitor kendaraannya pada gawai masing-masing untuk keamanan\cite{fais2014pengembangan}.


\section{Masalah}
\begin{enumerate}
	\item Membaca slot parkir yang masih kosong.
	\item Data tersebut dikirimkan ke firebase.
\end{enumerate}


\section{Tujuan dan Kontribusi}
\subsection{Tujuan}
\begin{enumerate}
	\item Alat ini bertujuan untuk memudahkan pengguna kendaraan dapat menemukan lahan parkir yang masih kosong.
	\item Pengguna dapat memonitor sehingga kendaraan tetap aman.
\end{enumerate}

\subsection{Kontribusi}
\begin{enumerate}
	\item Membangun alat yang dapat membaca slot parkir yang masih kosong.
	\item Membangun aplikasi yang dapat memonitoring dimana mobil kita di parkirkan.
\end{enumerate}

\section{Ruang Lingkup}
\begin{enumerate}
	\item Alat hanya dapat membaca slot parkir mana yang kosong dan mengirimkan statusnya ke firebase.
	\item Aplikasi hanya dapat membaca status kosong atau terisi dan menscan qrcode untuk memetakkan bahwasannya mobil di parkirkan ditempat tersebut.
\end{enumerate}